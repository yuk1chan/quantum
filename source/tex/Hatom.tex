\section{水素原子}
水素原子について考える。水素原子は$1$つの陽子と$1$つの電子から構成されている。
つまり$2$体問題だ。しかし、重心運動を分離することにより$1$体問題に帰着することができると\ref{CM_separation}でやった。
\subsection{水素原子のシュレディンガー方程式}
陽子の質量を$m_p$、電子の質量を$m_e$として、換算質量$\mu = \frac{m_p m_e}{m_p + m_e}$と陽子と電子間に働くクーロン力のポテンシャル$V(r) = -\frac{e^2}{r}$を用いると
シュレディンガー方程式は
\begin{equation}
	\left[ - \dfrac{\hbar}{2\mu}\Delta_r + V(r)\right]\psi(\bm{r})
	= \left[\dfrac{\hat{p_r}^2}{2\mu} + \dfrac{\hat{L}^2}{2\mu r^2} - \dfrac{e^2}{r}\right]\psi(\bm{r}) = E\psi(\bm{r})
\end{equation}
と書ける。$\psi(\bm{r}) = R_l(r)Y_l^m(\theta,\varphi)$と変数分離すると、動径方程式は
\begin{equation}
	\left[ -\dfrac{\hbar^2}{2\mu}\left(\dfrac{1}{r}\dfrac{d^2}{dr^2}r\right)  + \dfrac{l(l+1)\hbar^2}{2\mu r^2} - \dfrac{e^2}{r}\right]R_l(r) = ER_l(r)
\end{equation}
となる。ここで、$-\frac{2\mu}{\hbar}$を両辺にかけて整理すると、
\begin{equation}
	\label{R_eq}
	\dfrac{d^2 R_l}{dr^2} + \dfrac{2}{r}\dfrac{dR_l}{dr} + \dfrac{2\mu}{\hbar^2}\left[ E + \dfrac{e^2}{r} - \dfrac{l(l+1)\hbar^2}{2\mu r^2} \right]R_l = 0
\end{equation}
$r$が大きいところでは漸近的に
\begin{equation}
	\dfrac{d^2 R_l}{dr^2} = - \dfrac{2\mu E}{\hbar^2}R_l
\end{equation}
となり、クーロン力によって束縛されているので、$E < 0$である。
つまり、$r \to \infty$で
\begin{equation}
	R_l(r) \approx \exp\left(- \sqrt{\dfrac{2\mu |E|}{\hbar^2}}\right)
\end{equation}
となる。
ここで、無次元の独立変数
\begin{equation}
	\rho = \sqrt{\dfrac{8\mu|E|}{\hbar^2}}r
\end{equation}
を導入する。
まず、(\ref{R_eq})を$\dfrac{8\mu|E|}{\hbar^2}$で割ると、
\begin{equation}
	\dfrac{d^2 R_l}{d \frac{8\mu|E|}{\hbar^2} r^2} + \dfrac{2}{\sqrt{\frac{8\mu|E|}{\hbar^2}}r}\dfrac{dR_l}{d\sqrt{\frac{8\mu|E|}{\hbar^2}}r}
	 + \dfrac{1}{4|E|}\left[ E + \dfrac{e^2}{r} - \dfrac{l(l+1)\hbar^2}{2\mu r^2} \right]R_l = 0
\end{equation}
$\rho$を使って整理すると、
\begin{equation}
	\dfrac{d^2 R_l}{d \rho^2} + \dfrac{2}{\rho}\dfrac{d R_l}{d\rho} - \dfrac{l(l+1)}{\rho^2}R_l + \left(\dfrac{1}{4|E|}\dfrac{e^2}{r} + \dfrac{1}{4}\right)R_l = 0
\end{equation}
ここで、無次元の定数
\begin{equation}
	\label{lambda}
	\lambda = \dfrac{e^2}{\hbar}\sqrt{\dfrac{\mu}{2|E|}}
\end{equation}
を導入すると、
\begin{equation}
	\label{H_atom_R_eq}
	\dfrac{d^2 R_l}{d \rho^2} + \dfrac{2}{\rho}\dfrac{d R_l}{d\rho} - \dfrac{l(l+1)}{\rho^2}R_l + \left(\dfrac{\lambda}{\rho} - \dfrac{1}{4}\right)R_l = 0
\end{equation}
と書くことができる。
ここで、$\rho \to \infty$と$\rho \to 0$の極限を考えて解見つけていく。

$\rho \to \infty$の時、
\begin{equation}
	\dfrac{d^2 R_l}{d \rho^2} = \dfrac{1}{4}R_l
\end{equation}
となる。これの解は$R_l = Ae^{\pm\frac{1}{2}\rho}$となる($A$は任意定数)。しかし、$\rho \to \infty$の極限を考えているので、$Ae^{+\frac{1}{2}\rho}$は発散してしまう。
なので、解は$R_l = Ae^{-\frac{1}{2}\rho}$を採用する。
ここで、定数変化法ちっくに定数$A$が$\rho$の関数$A_l(\rho)$であるとする。\footnote{$\rho \to 0$の極限を考えるため}
これを(\ref{H_atom_R_eq})に代入して整理すると
\begin{equation}
	\label{A_l_eq}
	\dfrac{d^2 A_l}{d\rho^2} + \left( \dfrac{2}{\rho} - 1\right)\dfrac{d A_l}{d\rho} + \left(\dfrac{\lambda-1}{\rho} - \dfrac{l(l+1)}{\rho}\right)A_l = 0
\end{equation}
ここで、\ref{R_equation}で分かるように、$\lim_{r \to 0}V(r)r^2 = 0$を満たすポテンシャルの場合、$\rho$が小さい時は$R \sim \rho^l$、つまり、$A_l \sim \rho^l$と振る舞う。
なので、
\begin{equation}
	A_l(\rho) = \rho^l L(\rho)
\end{equation}
の形の解を求める。
これを(\ref{A_l_eq})に代入して$\rho^{l-1}$で割って整理すると
\begin{equation}
	\label{L_eq}
	\rho \dfrac{d^2 L}{d\rho^2} + (2l + 2 - \rho)\dfrac{dL}{d\rho} + (\lambda -1 -l)L = 0
\end{equation}
ここで、$L$が原点付近で級数展開できる
\begin{equation}
	\label{L_polynomials}
	L(\rho) = \sum_{n = 0}^\infty a_n\rho^n
\end{equation}
とする。これを微分すると、
\begin{equation}
	\dfrac{d L(\rho)}{d \rho} = \sum_{n = 1}^\infty a_n n \rho^{n-1} = \sum_{n = 0}^\infty a_n n \rho^{n-1}
\end{equation}
\begin{equation}
	\dfrac{d^2 L(\rho)}{d \rho^2} = \sum_{n = 2}^\infty a_{n}n(n-1)\rho^{n-2}
	= \sum_{n = 1}^\infty a_{n+1}n(n+1)\rho^{n-1} = \sum_{n = 0}^\infty a_{n+1}n(n+1)\rho^{n-1}
\end{equation}
なので、これを(\ref{L_eq})に代入すると、
\begin{equation}
	\sum_{n = 0}^\infty a_{n+2}(n+1)(n+2)\rho^{n+1} + (2l+2-\rho)\sum_{n = 0}^\infty a_{n+1}(n+1)\rho^n + (\lambda - 1 -l)\sum_{n = 0}^\infty a_n\rho^n = 0
\end{equation}
整理すると、
\begin{equation}
	\sum_{n = 0}^\infty[n(n+1)a_{n+1} + (\lambda -1 - l -n)a_n]\rho^n + \sum_{n = 0}^\infty[2l + 2]a_n n\rho^{n-1} = 0
\end{equation}
ここで、第$2$項をずらすと、
\begin{equation}
	\sum_{n = 0}^\infty[n(n+1)a_{n+1} + (\lambda -1 - l -n)a_n]\rho^n + \sum_{n = 0}^\infty[2l + 2]a_{n+1} (n+1)\rho^{n} = 0
\end{equation}
$\rho^n$の項をまとめると、
\begin{equation}
	\sum_{n = 0}^\infty[(n+1)(n+2l+2)a_{n+1} + (\lambda-1-l-n)a_n]\rho^n = 0
\end{equation}
$\rho^n$の係数は$0$なので、
\begin{equation}
	\label{an+1/an}
	\dfrac{a_{n+1}}{a_n} = \dfrac{n+l+1-\lambda}{(n+1)(n+2l+2)}
\end{equation}
従って、$n$が大きい時、$\frac{a_{n+1}}{a_n} \sim \frac{1}{n}$となる。
このまま$n$が大きくなり続ける、つまり、級数が無限に続くと、$L(\rho)\approx e^\rho$となるので、
$R_l \sim \rho^l e^\frac{\rho}{2}$となってしまい、$\rho\to\infty$で発散する。
つまり、級数$L$はどこかで終わらなければならない。
$n$が、ある$n_r$番目の項で$0$になれば$L$が有限の項で終わるので、
$n_r = \lambda - 1 -l$という条件を課してやれば良い。
つまり、$\lambda = n_r + l + 1$であり、正整数でなければならない。
\begin{equation}
	\lambda = n_r + l + 1 \equiv n
\end{equation}
$n_r$を動径量子数、$n$を主量子数と呼ぶ。
この$\lambda$の物理的な意味を考えよう。
(\ref{lambda})より
\begin{equation}
	\label{lambda_requ}
	\lambda = \dfrac{e^2}{\hbar}\sqrt{\dfrac{\mu}{2|E|}} = n
\end{equation}
これより
\begin{equation}
	\label{H_En}
	E_n = -|E_n| = -\dfrac{\mu e^4}{2\hbar^2 n^2}
\end{equation}
となる。つまり、エネルギーが量子化された。

\subsection{水素原子の波動関数}

(\ref{an+1/an})より、
\begin{align}
	a_k &= \dfrac{(k+l-\lambda)}{k(k+2l+1)}a_{k-1} \\
			&= (-1)\dfrac{(\lambda-l-k)}{k(k+2l+1)}a_{k-1} \\
			&= (-1)^2\dfrac{(\lambda-l-k)}{k(k+2l+1)}\dfrac{(\lambda-l-k+1)}{(k-1)(k+2l)}a_{k-2} \\
			&= (-1)^k\dfrac{(\lambda-l-k)}{k(k+2l+1)}\dfrac{(\lambda-l-k+1)}{(k-1)(k+2l)}
			\cdots\dfrac{(\lambda-l-2)}{2(2l+3)}\dfrac{(\lambda-l-1)}{1(2l+2)}a_0 \\
			&= (-1)^k\dfrac{(2l+1)!(\lambda-l-1)!}{k!(k+2l+1)!(\lambda-l-k-1)!}a_0
\end{align}
(\ref{lambda_requ})より$\lambda = n$なので、
\begin{equation}
	a_k = (-1)^k\dfrac{(2l+1)!(n-l-1)!}{k!(k+2l+1)!(n-l-k-1)!}a_0
\end{equation}
$k$は$0$から$n-l-1$の範囲である。
ここで、
\begin{equation}
	a_0 = \dfrac{((n+l)!)^2}{(n−l−1)!(2l+1)!}
\end{equation}
とすると、
\begin{equation}
	a_k = (-1)^{k}\dfrac{((n+l)!)^2}{k!(k+2l+1)!(n-l-k-1)!}
\end{equation}
よって、(\ref{L_polynomials})より
\begin{equation}
	L(\rho) = \sum_{k=0}^{n-l-1}(-1)^{k}\dfrac{((n+l)!)^2}{k!(k+2l+1)!(n-l-k-1)!}\rho^k
\end{equation}
これは、
\begin{equation}
	L_p^q(\rho) = \sum_{k = 0}^{p-q} (-1)^{k} \dfrac{p!p!}{k!(k+q)!(p-k-q)!}\rho^k
\end{equation}
と定義されるLaguerre陪多項式の$p = n+l,q = 2l+1$に他ならない。

よって、波動関数の動径部分は$e^{-\frac{\rho}{2}}\rho^l L_{n+l}^{2l+1}(\rho)$の形をしている。
規格化定数は、積分
\begin{equation}
	\int_0^\infty e^\rho \rho^{2l} [L_{n+l}^{2l+1}(\rho)]^2 \rho^2 d\rho
	= \dfrac{2n[(n+l)!]^3}{(n-l-1)!}
\end{equation}
をすると求まり、
規格化された水素原子の波動関数は
\begin{equation}
	\psi_{nlm}(r,\theta,\varphi) = R_{nl}(r)Y_l^m(\theta,\varphi)
\end{equation}
ただし、
\begin{align}
	R_{nl} = - \left\{ \left(\dfrac{2}{na_0}\right)^3\dfrac{(n-l-1)!}{2n[(n+l)!]^3}\right\}^{\frac{1}{2}}e^{-\frac{1}{2}\rho}\rho^l L_{n+l}^{2l+1}(\rho) \\
	a_0 = \dfrac{\hbar^2}{\mu e^2},~~~~\rho = \dfrac{2}{n a_0}r
\end{align}
である。

これを用いると、
エネルギー準位(\ref{H_En})は
\begin{equation}
	E_n = -\dfrac{e^2}{2a_0n^2}
\end{equation}
と書くことができる。

$n = 1$の時が水素原子の基底状態のエネルギーで
\begin{equation}
	E_1 = -\dfrac{e^2}{2a_0} = -13.6 \si{eV}
\end{equation}

ここで、$a_0 = \dfrac{\hbar^2}{\mu e^2} = 0.53 \text{\AA}$はBohr半径と呼ばれている。
