\section{極座標による3次元のシュレディンガー方程式(球対称なポテンシャル)}
球対称なポテンシャル$V(x,y,z) = V(r)$の中を運動している粒子の定常状態のシュレディンガー方程式は
\begin{equation}
	\left[ -\dfrac{\hbar^2}{2m}\nabla^2 + V(r) \right ]\psi(x,y,z) = E \psi(x,y,z)
\end{equation}
ポテンシャルが球対称な場合、直交座標ではなく、極座標で考えた方が解きやすい。
\begin{align}
	x &= r\sin\theta\cos\varphi \\
	y &= r\sin\theta\sin\varphi \\
	z &= r\cos\theta \\
	r^2 &= x^2 + y^2 + z^2 \\
	0 \leq r \leq \infty, 0 &\leq \theta \leq \pi, 0 \leq \varphi \leq 2\pi
\end{align}
の関係を用いて、
\begin{equation}
	\label{polar_schrodinger_eq}
	-\dfrac{\hbar^2}{2m}\left[
	\dfrac{1}{r}\dfrac{\partial^2}{\partial r^2}r
	+ \dfrac{1}{r^2\sin\theta}\dfrac{\partial}{\partial \theta}(\sin\theta)\dfrac{\partial}{\partial \theta}
	+ \dfrac{1}{r^2\sin^2\theta}\dfrac{\partial^2}{\partial\varphi^2}
	\right]\psi(r,\theta,\varphi)
	+ V(r)\psi(r,\theta,\varphi)
	= E\psi(r,\theta,\varphi)
\end{equation}
と書くことができる。複雑そうに見えるが変数分離形である。
これを、$r$を含む部分と、$\theta$と$\varphi$を含む部分に分けるために、
\begin{equation}
	\label{RY}
	\psi(r,\theta,\varphi) = R(r)Y(\theta,\varphi)
\end{equation}
とおく。これを(\ref{polar_schrodinger_eq})に代入して、$-\dfrac{2mr^2}{\hbar^2}$をかけると
\begin{equation}
	r\dfrac{\partial^2}{\partial r^2}(rR)Y + \left[
	\dfrac{1}{\sin\theta}(\sin\theta\dfrac{\partial}{\partial\theta})
	+ \dfrac{1}{\sin^2\theta}\dfrac{\partial^2}{\partial\varphi^2}
	\right]RY + \dfrac{2mr^2}{\hbar^2}(E-V(r))RY = 0
\end{equation}
両辺を$RY$で割ると、
\begin{equation}
	\dfrac{r}{R}\dfrac{\partial^2}{\partial r^2}(rR) + \dfrac{2mr^2}{\hbar^2}(E-V(r))
	= -\dfrac{1}{Y}\left[
	\dfrac{1}{\sin\theta}(\sin\theta\dfrac{\partial Y}{\partial\theta})
		+ \dfrac{1}{\sin^2\theta}\dfrac{\partial^2 Y}{\partial\varphi^2}
	\right]
\end{equation}
となる。左辺は$r$について、右辺は$\theta$と$\varphi$について関係しているので、両辺は定数でなければならない。この定数を$\lambda$とおく。

動径部分の方程式として
\begin{equation}
	\label{radius_eq}
	\dfrac{1}{r^2}\dfrac{d^2}{d r^2}(rR) + \dfrac{2m}{\hbar^2}(E-V(r) - \dfrac{\lambda}{r^2})R = 0
\end{equation}
角部分の方程式として
\begin{equation}
	\label{angle_eq}
	\dfrac{1}{\sin\theta}(\sin\theta\dfrac{\partial Y}{\partial\theta})
		+ \dfrac{1}{\sin^2\theta}\dfrac{\partial^2 Y}{\partial\varphi^2} +\lambda Y = 0
\end{equation}
を得る。角部分の方程式はポテンシャル$V(r)$にも、エネルギー$E$にも関係がないことが分かる。
また、角部分の方程式も変数分離形なので、
\begin{equation}
	Y(\theta,\varphi) = \Theta(\theta)\Phi(\varphi)
\end{equation}
とおくと、(\ref{angle_eq})に代入して整理すると、
\begin{equation}
	\sin\theta\dfrac{\partial}{\partial\theta}(\sin\theta\dfrac{d\Theta(\theta)}{d\theta})\dfrac{1}{\Theta(\theta)} + \lambda\sin^2\theta
	= -\dfrac{1}{\Phi(\varphi)}\dfrac{d\Phi(\varphi)}{d\varphi}
\end{equation}
を得る。両辺は定数でなければならない。この定数を$m^2$とおく\footnote{粒子の質量ではない。}と、
\begin{equation}
	\label{Phi_eq}
	\dfrac{1}{\Phi(\varphi)}\dfrac{d\Phi(\varphi)}{d\varphi} m^2\Phi(\varphi)= 0
\end{equation}
\begin{equation}
	\label{Theta_eq}
	\dfrac{1}{\sin\theta}\dfrac{d}{d\theta}(\sin\theta\dfrac{d\Theta(\theta)}{d\theta}) +(\lambda - \dfrac{m^2}{\sin^2\theta})\Theta(\theta) = 0
\end{equation}
となる。

(\ref{Phi_eq})を解く。これは単振動と同じ形である。特殊解は
\begin{equation}
	\label{Phi_particular_solution}
	\Phi(\varphi) = e^{im\varphi}
\end{equation}
である。これは多価関数である。固有関数は一価関数であるという要請をおくと、
$\Phi(2\pi) = \Phi(0)$なので、特殊解(\ref{Phi_particular_solution})は$e^{2\pi im} = 1$
を満たさなければならない。
従って、
\begin{equation}
	m = 0, \pm 1, \pm 2, \cdots
\end{equation}
になる。$\int_0^{2\pi}\Phi^*\Phi d\varphi = 1$と規格化すると、
\begin{equation}
	\Phi_m(\varphi) = \dfrac{1}{\sqrt{2\pi}}e^{im\varphi},\quad (m = 0, \pm 1, \pm 2, \cdots)
\end{equation}
が(\ref{Phi_eq})の解である。

次に(\ref{Theta_eq})を解く。
$z = \cos\theta$と変数変換し、
\begin{equation}
	\Theta(\theta) \equiv P^m(z)
\end{equation}
とおくと、(\ref{Theta_eq})は、
\begin{equation}
	\frac{1}{dz}\left[ (1-z^2)\dfrac{dP^m}{dz}\right] + \left[ \lambda - \dfrac{m^2}{1-z^2}\right]P^m = 0
\end{equation}
となる。$ 0 \leq \theta \leq \pi$より、$-1 \leq z \leq 1$である。

$m = 0$の時、これはlegendreの微分方程式(\ref{legendre_DE})である。
このことから、$\lambda = l(l+1)$と分かる。($l$は非負整数)

$m \neq 0$の時、legendreの微分方程式(\ref{legendre_DE})の解(\ref{legendre_polynomials})$P_l$を用いて、
\begin{equation}
	\label{general_legendre_function}
	P_l^m = (1-z^2)^{\frac{|m|}{2}}\dfrac{d^{|m|}P_l(z)}{dz^{|m|}}
\end{equation}
となる。$P_l$は$l$次多項式なので、$|m| \leq l$である

(\ref{general_legendre_function})をLegendreの陪関数と呼ぶ
\subsection{球面調和関数}
(\ref{Theta_eq})で$\lambda = l(l+1)$とした時の解は、(\ref{Phi_particular_solution})、(\ref{general_legendre_function})と規格化定数$N_{lm}$を使って、
\begin{equation}
	Y_l^m(\theta,\varphi) = N_{lm} P_l^m(\cos\theta)\Phi_m(\varphi)
\end{equation}
となる。これを球面調和関数と呼ぶ。

規格化された球面調和関数は
\begin{equation}
	\label{spherical_harmonics}
Y_l^m(\theta,\varphi) = \epsilon\sqrt{\dfrac{2l+1}{4\pi}\dfrac{(l-|m|)!}{(l+|m|)!}}P_l^m(\cos\theta)\Phi_m(\varphi)
\end{equation}
ただし、
\begin{equation}
	\epsilon =
	\begin{cases}
		(-1)^m  & (m > 0) \\
		1 &  (m \leq 0)
	\end{cases}
\end{equation}
また、
\begin{equation}
	\int_0^{2\pi}d\varphi\int_0^\pi (Y_l^m)^*Y_{l'}^{m'}\sin\theta d\theta
	= \delta_{ll'}\delta_{mm'}
\end{equation}
と、直交関係を満たす。

$Y_l^m$をいくつか書いておく。
\begin{align}
	Y_0^0 &= \dfrac{1}{\sqrt{4\pi}} \\
	Y_1^0 &= \sqrt{\dfrac{3}{4\pi}}\cos\theta \\
	Y_1^{\pm1} &= \mp\sqrt{\dfrac{3}{8\pi}}\sin\theta e^{\pm i\varphi} \\
	Y_2^0 &= \sqrt{\dfrac{5}{4\pi}}\left( \dfrac{3}{2} \cos^2\theta - \dfrac{1}{2}\right) \\
	Y_2^{\pm1} &= \pm \sqrt{\dfrac{15}{8\pi}}\sin\theta\cos\theta e^{\pm i\varphi} \\
	Y_2^{\pm2} &= \sqrt{\dfrac{15}{32\pi}}\sin^2\theta e^{\pm2i\varphi}
\end{align}

\subsection{角運動量状態}
古典力学では角運動量$\bm{L}$は運動量$\bm{p}$を用いて
\begin{equation}
	\bm{L} = \bm{r} \times \bm{p}
\end{equation}
量子力学では角運動量$\bm{\hat{L}}$は運動量$\bm{\hat{p}} = -i\hbar\nabla$を用いて
\begin{equation}
	\bm{\hat{L}} = \bm{\hat{r}} \times \bm{\hat{p}}
\end{equation}
と表せる。成分ごとに書くと
\begin{align}
	\hat{L_x} &= \hat{y}\hat{p_z} - \hat{z}\hat{p_y} = -i\hbar\left( y\dfrac{\partial}{\partial z} - z\dfrac{\partial}{\partial y}\right) \\
	\hat{L_y} &= \hat{z}\hat{p_x} - \hat{x}\hat{p_z} = -i\hbar\left( z\dfrac{\partial}{\partial x} - x\dfrac{\partial}{\partial z}\right) \\
	\hat{L_z} &= \hat{x}\hat{p_y} - \hat{y}\hat{p_x} = -i\hbar\left( x\dfrac{\partial}{\partial y} - y\dfrac{\partial}{\partial x}\right)
\end{align}
極座標に変換すると、
\begin{align}
	\hat{L_x} &= i\hbar\left( \sin\varphi\dfrac{\partial}{\partial \theta} + \dfrac{\cos\theta\cos\varphi}{\sin\theta}\dfrac{\partial}{\partial \varphi} \right) \\
	\hat{L_y} &= -i\hbar\left( \cos\varphi\dfrac{\partial}{\partial \theta} - \dfrac{\cos\theta\sin\varphi}{\sin\theta}\dfrac{\partial}{\partial \varphi} \right) \\
	\label{L_z}
	\hat{L_z} &= -i\hbar \dfrac{\partial}{\partial \varphi}
\end{align}
となる。$\bm{\hat{L}}^2 = \hat{L_x^2} + \hat{L_y^2} + \hat{L_z^2}$より、
\begin{equation}
	\label{L^2}
	\bm{\hat{L}}^2 = \hbar^2 \left[ \dfrac{1}{\sin\theta} \dfrac{\partial}{\partial \theta} \left( \sin\theta \dfrac{\partial}{\partial\theta} \right)
	+ \dfrac{1}{\sin^2\theta}\dfrac{\partial^2}{\partial\varphi^2} \right]
\end{equation}
また、ここで、
\begin{equation}
	\label{p_r}
	\hat{p_r} = -i\hbar\dfrac{1}{r}\dfrac{\partial}{\partial r}r
\end{equation}
を用意する。これは動径方向の運動量演算子と呼ばれる。
これを$2$乗すると、
\begin{equation}
	\label{p_r^2}
	\hat{p_r^2} = -\hbar^2 \dfrac{1}{r}\dfrac{\partial^2}{\partial r^2}r
\end{equation}
となる。
よって、(\ref{L^2})と(\ref{p_r^2})より、
\begin{equation}
	\label{p_rL^2}
	\dfrac{\hat{p_r^2}}{2m} + \dfrac{ \bm{\hat{L}}^2 }{2mr^2} = -\dfrac{\hbar^2}{2m}\nabla^2
\end{equation}

ここで、(\ref{angle_eq})と(\ref{L^2})より、
\begin{equation}
	\label{L^2Y}
	\hat{L^2}Y_l^m(\theta,\varphi) = l(l+1)\hbar^2 Y_l^m(\theta,\varphi)
\end{equation}

また、(\ref{L_z})と(\ref{Phi_particular_solution})より、
\begin{equation}
	\label{L_zY}
	\hat{L_z}Y_l^m(\theta,\varphi) = m\hbar Y_l^m(\theta,\varphi)
\end{equation}
が得られる。

球関数$Y_l^m(\theta,\varphi)$は、固有値$l(l+1)\hbar^2$をもつ角運動量の$2$乗の固有関数であり、
同時に固有値$m\hbar$をもつ角運動量の$z$成分の固有関数でもある。

(\ref{L^2Y})に出てくる量子数$l$を軌道角運動量量子数、
(\ref{L_zY})に出てくる量子数$m$を軌道磁気量子数という。

(\ref{p_rL^2})を用いて、極座標のシュレディンガー方程式(\ref{polar_schrodinger_eq})は
\begin{equation}
	\left[ \dfrac{\hat{p_r^2}}{2m} + \dfrac{ \bm{\hat{L}}^2 }{2mr^2} + V(r)\right]\psi = E\psi
\end{equation}
と書くことができる。
これは、極座標におけるハミルトニアン$H$は
\begin{equation}
	\label{polar_H}
	H = \dfrac{\hat{p_r^2}}{2m} + \dfrac{ \bm{\hat{L}}^2 }{2mr^2} + V(r)
\end{equation}
であることが分かる。

ハミルトニアン$H$と角運動量演算子の交換関係は
\begin{equation}
	[\hat{H},\hat{L^2}] = [\hat{H},\hat{L_z}] = [\hat{L^2},\hat{L_z}] = 0
\end{equation}
である。
ハミルトニアンが(\ref{polar_H})である系は、エネルギー$E$と角運動量の$2$乗$\hat{L^2}$と角運動量の$z$成分$\hat{L_z}$の値が決まっている定常状態を持つ。
この定常状態の波動関数は$3$つの演算子$\hat{H}、\hat{L^2}、\hat{L_z}$の全ての同時固有状態である。

\subsection{動径方向の運動量演算子}
(\ref{p_r})で動径方向の運動量演算子を定義した。この演算子と位置の演算子$\hat{r}$との交換子を考える。
\begin{align}
	[\hat{r},\hat{p_r}]
	&= -i\hbar r\dfrac{\partial}{\partial r} + i\hbar\dfrac{1}{r}\dfrac{\partial}{\partial r}r^2 \\
	&= -i\hbar\left(\dfrac{\partial}{\partial r} + r\dfrac{\partial}{\partial r}\right)
	+ i\hbar\dfrac{1}{r}\left(2r + r^2\dfrac{\partial}{\partial r}\right) \\
	&= i\hbar
\end{align}
ここで、
\begin{equation}
	\label{p_r_tmp}
	\hat{p_r} = -i\hbar\dfrac{\partial}{\partial r}
\end{equation}
と、定義しても
\begin{align}
	[\hat{r},\hat{p_r}]
	&= -i\hbar r\dfrac{\partial}{\partial r} + i\hbar\dfrac{\partial}{\partial r}r \\
	&= -i\hbar r\dfrac{\partial}{\partial r} + i\hbar + i\hbar r\dfrac{\partial}{\partial r} \\
	&= i\hbar
\end{align}
と、交換関係$[\hat{r},\hat{p_r}] = i\hbar$を満たす。
つまり、交換関係だけでは動径$r$に共役な運動量$p_r$を決めることができない。
決めるためには、エルミート演算子が満たすべき内積の関係式$(\psi^*,p_r\psi)=((\psi p_r)^*,\psi)$
を考えれば良い。

まずは、(\ref{p_r_tmp})で先程の内積の関係式を考えてみる。
\begin{align}
	(\psi^*,p_r\psi) &=
	\int_0^\pi d\theta \int_0^{2\pi} d\varphi \int_0^\infty r^2dr \psi^*\dfrac{\hbar}{i}\dfrac{\partial \psi}{\partial r} \\
	&= \int_0^\pi d\theta \int_0^{2\pi} d\varphi \times
	\left[
		\dfrac{\hbar}{i}[|r^2\psi|^2]_0^\infty - \int_0^\infty dr \dfrac{\hbar}{i} \dfrac{\partial}{\partial r}(r^2\psi^*)\psi
	\right]\\
	&= \int_0^\pi d\theta \int_0^{2\pi} d\varphi \times
	\left[
		\dfrac{\hbar}{i}[|r^2\psi|^2]_0^\infty - \int_0^\infty dr \dfrac{\hbar}{i} 2r|\psi|^2
		+ \int_0^\infty r^2dr\left( \dfrac{\hbar}{i}\dfrac{\partial}{\partial r}\psi\right)^*\psi
	\right]\\
	&= \int_0^\pi d\theta \int_0^{2\pi} d\varphi \times
	\left[
		\dfrac{\hbar}{i}[|r\psi|^2]_0^\infty - \int_0^\infty dr \dfrac{\hbar}{i} 2r|\psi|^2 \right] + ((\psi p_r)^*,\psi)
\end{align}
よって、
\begin{equation}
	(\psi^*,p_r\psi)-((\psi p_r)^*,\psi) = \int_0^\pi d\theta \int_0^{2\pi} d\varphi \times
	\left[
		\dfrac{\hbar}{i}[|r\psi|^2]_0^\infty - \int_0^\infty dr \dfrac{\hbar}{i} 2r|\psi|^2 \right]
\end{equation}
となり、右辺が$0$になれば$\hat{p_r}$はエルミート演算子である。
波動関数$\psi$が規格化できれば無限遠で$0$になる。
\begin{equation}
	\label{psi_require1}
	\lim_{r \to \infty} r\psi = 0
	\footnote{$\psi$ではなく$r\psi$なのかというと、$r\psi$を波動関数と見ることで動径方程式を考えるときに役立つため(?)}
\end{equation}
なので、今、
\begin{equation}
	(\psi^*,p_r\psi)-((\psi p_r)^*,\psi) = \int_0^\pi d\theta \int_0^{2\pi} d\varphi \times
\left[
	-\dfrac{\hbar}{i}\left.|r^2\psi|^2\right|_{r=0} - \int_0^\infty dr \dfrac{\hbar}{i} 2r|\psi|^2
\right]
\end{equation}
ここで、
\begin{align}
	\left.|r^2\psi|^2\right|_{r=0} &\geq 0 \\
	r|\psi|^2 &\geq 0
\end{align}
なので、
\begin{equation}
	(\psi^*,p_r\psi)-((\psi p_r)^*,\psi) = \int_0^\pi d\theta \int_0^{2\pi} d\varphi \times
\left[
	-\dfrac{\hbar}{i}(0以上) - \int_0^\infty \dfrac{\hbar}{i} (0以上) dr
\right]
\end{equation}
と分かるので、[~]の中身が$0$になることはない。
なので、(\ref{p_r_tmp})はエルミートではないことが分かる。
となると、(\ref{p_r})がエルミートである必要がある。
同様に内積を計算し、波動関数が規格化できるということから無限遠で$0$になることを課すと
\begin{equation}
	(\psi^*,p_r\psi)-((\psi p_r)^*,\psi) = \int_0^\pi d\theta \int_0^{2\pi} d\varphi \times
\left[
	-\dfrac{\hbar}{i}\left.|r^2\psi|^2\right|_{r=0}
\right]
\end{equation}
となる。
なので、
\begin{equation}
	\label{psi_boundary_reauire}
	\lim_{r \to 0} r\psi = 0
\end{equation}
が成立すれば(\ref{p_r})はエルミート演算子であり、動径$r$に共役な運動量演算子である。
これが波動関数$\psi$が原点で満たすべき条件となる。


\subsection{動径方程式}\label{R_equation}

今、任意の球対称なポテンシャルの中で、$L^2$と$L_z$がきまった値をもつ粒子の定常状態の波動関数は
(\ref{RY})と(\ref{spherical_harmonics})より
\begin{equation}
	\psi(r,\theta,\varphi) = R_l(r)Y_l^m(\theta,\varphi)
\end{equation}
となる。$Y_l^m$が$L^2$の固有関数なので$\lambda = l(l+1)$で、(\ref{radius_eq})より、$R_l$は
\begin{equation}
	\label{R_l}
	\dfrac{1}{r^2}\dfrac{d^2}{d r^2}(rR_l(r)) + \dfrac{2m}{\hbar^2}(E-V(r) - \dfrac{l(l+1)}{r^2})R_l(r) = 0
\end{equation}
の解である。これをシュレディンガー方程式に似た形に変形したい。
そのためには、
\begin{equation}
	R_l(r) = \dfrac{\chi_l(r)}{r}
\end{equation}
と置いて代入して整理すると、
\begin{equation}
	\label{R_l_Veff}
	-\dfrac{\hbar^2}{2m}\dfrac{d^2 \chi_l(r)}{d r^2} + \left[V(r) +\dfrac{l(l+1)}{2mr^2}\right]\chi_l(r) = E\chi_l(r)
\end{equation}

これは、ポテンシャルエネルギー
\begin{equation}
	\label{Veff}
	V_{eff}(r) = V(r) + \dfrac{l(l+1)}{2mr^2}
\end{equation}
の中$(0 \leq r \leq \infty)$で運動する$1$次元のシュレディンガー方程式と同じ形である。

このポテンシャルの右辺第$2$項について考えてみよう。
古典的な描像では、電子は原子核の周りを円運動していると見ることができる。
そのため、遠心力が働いている。この遠心力は
\begin{equation}
	F = mr\omega^2 = m\dfrac{v^2}{r} = \dfrac{p^2}{mr} = \dfrac{(rp)^2}{mr^3} = \dfrac{L^2}{mr^3}
\end{equation}
この時のポテンシャルエネルギーは
\begin{equation}
	U = - \int_\infty^r F dr' = -\int_\infty^r \dfrac{L^2}{mr'^3} dr'
	= \left[ \dfrac{L^2}{2mr'^2} \right]_\infty^r = \dfrac{L^2}{2mr^2}
\end{equation}
となる。
角運動量の大きさ$|L|$が$l(l+1)\hbar$に等しい時、(\ref{Veff})の右辺の第$2$項に等しくなり、$l>0$の粒子が原点に近づくことを妨げる働きをすることが分かる。このことから、遠心力ポテンシャルと呼ばれる。

「$1$次元問題では、離散スペクトルの現れるエネルギー準位は縮退していない」という性質が成立するので、
(\ref{R_l_Veff}の解である波動関数の動径部分$R_l$はエネルギーを与えると完全に決まる。
また、各部分は$l$と$m$の値によって完全に決まるので、球対称なポテンシャルの中を運動する粒子の波動関数は
$E$と$l$と$m$の$3$つの値によって完全に決定できる。
ただ、(\ref{R_l})は量子数$m$を含んでいないので、$E$と$l$だけを決めても、$(2l+1)$個の$m$の値に対応して$(2l+1)$重に縮退している。($m\leq l$かつ$m = 0, \pm1,\pm2,\pm3,\cdots$なので)

また、波動関数を規格化条件は
\begin{align}
	\int_0^\infty	dr \int_0^\pi d\theta \int_0^{2\pi} d\varphi |\psi{r,\theta,\varphi}|^2 r^2 \sin\theta
	&= \int_0^\infty dr \int_0^\pi d\theta \int_0^{2\pi} d\varphi |RY|^2 r^2 \sin\theta \\
	&= \int_0^\infty dr |R|^2 r^2 \\
	&= \int_0^\infty dr |\chi_l|^2
\end{align}
と、規格化条件も$1$次元のと同じ性質である。この積分は有限になるため、波動関数は無限遠で
\begin{equation}
	\lim_{r \to \infty} \chi(r) = \lim_{r \to \infty} rR(r) = 0
\end{equation}
となる。つまり、$R$は$r^{-1}$より速く$0$に収束しなければならないことが分かる。

今、波動関数$\psi = R_lY_l^m$に課される条件は(\ref{psi_boundary_reauire})である。
動径$r$に依存する部分だけを見れば
\begin{equation}
	\label{normalization_requaire}
	\lim_{r \to 0}rR_l(r) = \lim_{r \to 0}\chi_l(r) = 0
\end{equation}
と分かる。
つまり、$\chi_l(r)$の原点での振る舞いを調べることで、$\psi$が本当にシュレディンガー方程式(\ref{polar_schrodinger_eq})の解になっているかどうかを確かめられる。
今、$l \neq 0$として
波動関数$\chi_l$とポテンシャル$V(r)$が原点付近で冪級数に展開できるとする。
\begin{equation}
	\chi_l(r) = r^\alpha\sum_{k=0}^\infty a_kr^k ~ (a_0 \neq 0)
\end{equation}
\begin{equation}
	V(r) = r^\beta\sum_{k'=0}^\infty V_{k'}r^{k'} ~ (\beta > -2) \footnote{$\lim_{r \to 0} V(r)r^2 = 0$となるポテンシャルを考えている。}
\end{equation}
これを(\ref{R_l_Veff})に代入して整理すると、
\begin{equation}
	\dfrac{\hbar^2}{2m}\sum_{k=0}^\infty a_k
	\left[
		-(\alpha+k)(\alpha+k-1) + \dfrac{l(l+1)}{2m}
	\right]r^{k+\alpha-2}
	+ \sum_{k=0}^\infty \sum_{k'=0}^\infty V_{k'}a_k r^{\alpha+\beta+k+k'}
	= E\sum_{k=0}^\infty a_k r^{k+\alpha}
\end{equation}
となる。ここで$r$の最低次の項を考える。

左辺の最低次の項は$k = 0$の時の$r^{\alpha-2}$の項である。というのも、ポテンシャル$V(r)$から生じる部分は
$k = k' = 0$の時$r^{\alpha+\beta}$となるが、$\beta > 2$という条件があるので、$r^{\alpha+\beta}$は$r^{\alpha-2}$より高次になるためだ。

右辺の最低次の項は$k = 0$の時の$r^\alpha$の項である。

この$k = k' = 0$時を考えると
\begin{equation}
	\dfrac{\hbar^2}{2m}a_0 [-\alpha(\alpha-1)+l(l+1)]r^{\alpha-2} + V_0a_0r^{\alpha+\beta} = Ea_0r^\alpha
\end{equation}
ここで、$r^{\alpha-2}$が最低次の項であるため、原点付近ではそれ以外の次数が高い項は無視できると考えられる。
つまり、
\begin{equation}
	\dfrac{\hbar^2}{2m}a_0 [-\alpha(\alpha-1)+l(l+1)]r^{\alpha-2} = 0
\end{equation}
従って、$a_0 \neq 0$なので
\begin{equation}
	\alpha(\alpha-1) = l(l+1)]r^{\alpha-2}
\end{equation}
これの解は$\alpha = -l$もしくは$\alpha = l+1$である。
ゆえに、
\begin{equation}
	\chi_l(r) \approx r^{-l},~もしくは~r^{l+1}
\end{equation}
$r^{-l}$は原点で発散してしまうので、規格化条件(\ref{normalization_requaire})を満たさない。
一方で、$r^{l+1}$は規格化条件(\ref{normalization_requaire})を満たす。
なので、$\alpha = l+1$の時、つまり、$\chi_l(r) \approx r^{l+1}$と振る舞う解のみが物理的に許される。
\begin{equation}
	\chi_l(0) = 0
\end{equation}
となる。
さて、これまでは$l \neq 0$の時を考えてきたが、$l = 0$の時はどうなるだろうか?

実は、$l=0$の時でも原点付近で$\chi_0(0) = 0$になる。
\begin{proof}
	$\chi_0(0) = c \neq 0$と仮定する。

	すると、原点付近で$R_0(r) = \frac{c}{r}$なので、これを$\psi = R_0Y_0^0$に代入すると
	\begin{equation}
		\psi = \dfrac{c}{\sqrt{4\pi}}\dfrac{1}{r}
	\end{equation}
	よって、
	\begin{align}
		\hat{H}\psi &= \left[-\dfrac{\hbar^2}{2m}\nabla^2 + V(r)\right]\psi \\
		&= \left[ -\dfrac{\hbar^2}{2m}\dfrac{2}{r^2} + V(r)\right]\psi
	\end{align}
	となる。しかし、シュレディンガー方程式は$\hat{H}\psi = E\psi$と$\psi$に関して斉次であるが、これは斉次ではない。
	仮定$\chi_0(0) = c \neq 0$によって矛盾が生じてしまったので、$\chi_0(0) = 0$である。
	\qed
\end{proof}

よって、以上のことから、
{\bf 原点における条件$\chi_l(r = 0)(l = 0,1,2,\cdots)$を$1$次元の動径方程式(\ref{radius_eq})に課して解くことで、
球対称なポテンシャルの中を運動している粒子の$3$次元シュレディンガー方程式を解くことと同等になる。}
ということがわかる。

ハミルトニアン(\ref{polar_H})は$2$個の固有値$\pm1$を持つ空間反転演算子$\hat{P}$と交換する。
空間反転演算子$\hat{P}$は
\begin{equation}
	\hat{P}\psi(\bm{r},t) = \psi(-\bm{r},t)
\end{equation}
と波動関数に作用すると定義される。
$\hat{P}$の演算に対して、動径$r$は不変だが、角度については$\theta \to \pi - \theta,~\varphi \to \varphi + \pi$
と変換される。

中心力ポテンシャル$V(r)$の中で質量$m$の粒子が束縛されているとする。この時、与えられた角運動量$l$の値に対して許される最低エネルギー固有値
$E_l^{(0)}$は、$l$の増大とともに大きくなる。
これは、角運動量があるとハミルトニアンに$\dfrac{l(l+1)\hbar^2}{2mr^2}$という正符号の項が付け加えられ、これが$l$とともに増大することに起因する。
ハミルトニアンを
\begin{equation}
	\label{H_l}
	\hat{H_l} = -\dfrac{\hbar^2}{2m}\dfrac{d^2}{dr^2} + \left[ V(r) + \dfrac{l(l+1)\hbar^2}{2mr^2} \right]
\end{equation}
とおき、この時の最低固有値を$E_l^{(0)}$、固有関数を$\chi_l^{(0)}$とおく。(\ref{H_l})より
\begin{equation}
	\hat{H_{l+1}} = \hat{H_l} + \dfrac{(l+1)\hbar^2}{mr^2}
\end{equation}
となるので、
\begin{align}
	E_{l+1}^{(0)}
	&= \int {\chi_{l+1}^{(0)}}^* \hat{H_{l+1}} \chi_{l+1}^{(0)} d^3r \\
	\label{chiHchi}
	&= \int {\chi_{l+1}^{(0)}}^* \hat{H_l} \chi_{l+1}^{(0)} d^3r + \int \dfrac{(l+1)\hbar^2}{mr^2} {\chi_{l+1}^{(0)}}^* \chi_{l+1}^{(0)} d^3r
\end{align}
ここで、$\hat{H_l}$の固有関数を$\chi_l^{(k)}$とおくと、$\hat{H_l}\chi_l^{(k)} = E_l^{(k)}\chi_l^{(k)}~(k\neq 0は励起状態)$

$\chi_{l+1}^{(0)}$を完全系$\{\chi_l^{(k)}\}$で展開すると、
\begin{equation}
	\chi_{l+1}^{(0)} = \sum_k c_k \chi_l^{(k)},~~~\sum_k |c_k|^2 = 1
\end{equation}
これを(\ref{chiHchi})の第$1$項に代入して計算すると
\begin{align}
	\int {\chi_{l+1}^{(0)}}^* \hat{H_l} \chi_{l+1}^{(0)} d^3r
	&= \sum_k |c_k|^2 E_l^{(k)}
\end{align}
ここから、最低固有値$E_l^{(0)}$を引くと
\begin{align}
	\sum_k |c_k|^2 E_l^{(k)} - E_l^{(0)}
	&= \sum_k |c_k|^2 E_l^{(k)} - E_l^{(0)}\sum_k|c_k|^2 \\
	&= \sum_k |c_k|^2(E_l^{(k)} - E_l^{(0)}) > 0
\end{align}
つまり、
\begin{equation}
	\int {\chi_{l+1}^{(0)}}^* \hat{H_l} \chi_{l+1}^{(0)} d^3r > E_l^{(0)}
\end{equation}
また、(\ref{chiHchi})の第$2$項が正であるので、
\begin{equation}
	E_{l+1}^{(0)} > E_{l}^{(0)}
\end{equation}
ということが分かる。
従って、
\begin{equation}
	E_{0}^{(0)} < E_{1}^{(0)} < E_{2}^{(0)} < \cdots < E_{l}^{(0)} < E_{l+1}^{(0)} < \cdots
\end{equation}
である。

粒子の角運動量$l$が様々な値をとる状態を表すために、以下のような慣用記号が使われる。
\begin{table}[htb]
	\begin{center}
	\begin{tabular}{cccccccc}
		$l=$     & $0$ & $1$ & $2$ & $3$ & $4$ & $5$ & $\cdots$ \\
		に対応して & $s$ & $p$ & $d$ & $f$ & $g$ & $h$ & $\cdots$
	\end{tabular}
	\end{center}
\end{table}

これより、中心力ポテンシャルの中を運動する粒子の基底状態は$s$状態であると分かる。
