
\section{無限に深い1次元井戸型ポテンシャル}
無限に深い$1$次元井戸型ポテンシャルというものを考えてみよう。これはポテンシャル$V(x)$が
\begin{equation}
  V(x) =
  \begin{cases}
    0,       & (-a < x < a) \\
    \infty , & (x \leq -a , a \leq x)
  \end{cases}
\end{equation}
となっている。
$x \leq -a , a \leq x$でポテンシャルが無限大なので、この範囲で粒子は存在できない。
そのため、$-a < x < a$の範囲で時間に依存しないシュレディンガー方程式
\begin{equation}
  \label{time_independent_schrodinger_eq_in_well}
  - \dfrac{\hbar^2}{2m} \dfrac{d^2 \psi(x)}{d x^2} = E\psi(x)
\end{equation}
を解けばよい。
ここで、$x \leq -a , a \leq x$の範囲で粒子が存在できないことから、
境界条件
\begin{equation}
  \label{boundary_req}
  \psi(a) = \psi(-a) = 0
\end{equation}
を課すことになる。
(\ref{time_independent_schrodinger_eq_in_well})を変形して
\begin{equation}
  \dfrac{d^2 \psi(x)}{d x^2} = -\dfrac{2mE}{\hbar^2}\psi(x)
\end{equation}
ここで、
\begin{equation}
  \label{k_equiv}
  k \equiv \dfrac{\sqrt{2mE}}{\hbar}
\end{equation}
とおけば
\begin{equation}
  \dfrac{d^2 \psi(x)}{d x^2} = -k^2\psi(x)
\end{equation}
となり、この微分方程式を解けばよい。
一般解は
\begin{equation}
  \psi(x) = A\cos(kx) + B\sin(kx)
\end{equation}
となる。($A$、$B$は任意定数である。)

ここで、境界条件(\ref{boundary_req})より
\begin{align}
  A\cos(ka) + B\sin(ka) &= 0 \\
  A\cos(kx) - B\sin(ka) &= 0
\end{align}
これより、
\begin{align}
  A\cos(ka) &= 0 \\
  B\sin(ka) &= 0
\end{align}
が得られる。この条件を満たす解として$A=B=0$があるが、この時の波動関数$\psi(x)$は恒等的に$0$になるので、物理的に意味のない解になる。(つまり、粒子が存在しない状態になる。)

物理的に意味のある解を求めていこう。

$A = 0$の時、$B \neq 0$より
\begin{equation}
  \sin(ka) = 0
\end{equation}
つまり、
\begin{equation}
  ka = \dfrac{(n+1)\pi}{2}, \quad(n = 1, 3, 5, \ldots)
\end{equation}

$B = 0$の時、$A \neq 0$より
\begin{equation}
  \cos(ka) = 0
\end{equation}
つまり、
\begin{equation}
  ka = \dfrac{(n+1)\pi}{2}, \quad(n = 0, 2, 4, \ldots)
\end{equation}

従って、
\begin{equation}
  k = \dfrac{(n+1)\pi}{2a}, \quad(n = 0, 1, 2, \ldots)
\end{equation}
となる。
ここで、(\ref{k_equiv})より
\begin{equation}
  E = \dfrac{(n+1)^2\pi^2\hbar^2}{8ma^2}, \quad(n = 0,1,2,\ldots)
\end{equation}
と固有エネルギー$E$が求まる。

さて、固有状態の波動関数は
\begin{align}
  \psi(x) &= A\cos{\dfrac{(n+1)\pi}{2a}x}, \quad(n = 0, 2, 4, \ldots) \\
  \psi(x) &= B\sin{\dfrac{(n+1)\pi}{2a}x}, \quad(n = 1, 3, 5, \ldots)
\end{align}
と求まる。$n$が偶数の場合は、波動関数も偶関数であり、$n$が奇数の場合は、波動関数も奇関数であることが分かる。

定数$A$、$B$は波動関数の規格化定数である。
\begin{equation}
  1=\int_{-a}^{a} A^{2} \cos ^{2} k x \mathrm{d} x=A^{2} a, \quad
  1=\int_{-a}^{a} B^{2} \sin ^{2} k x \mathrm{d} x=B^{2} a
\end{equation}
より、
\begin{equation}
  A = B = \dfrac{1}{\sqrt{a}}
\end{equation}
と求めることができる。
