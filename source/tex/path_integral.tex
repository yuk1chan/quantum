
\section{経路積分}
\subsection{時間推進演算子とTrotter公式}
Schr\"{o}dinger方程式によって、系の時刻$t$における状態$\ket{\psi(t)}$は
\begin{equation}
  i\hbar \dfrac{\partial}{\partial t}\ket{\psi(t)} = H\ket{\psi(t)}
  \label{schrodinger_eq1}
\end{equation}
となる。時間推進演算子(time evolution operator)$U(t,t_0)$を用いることで、$\ket{\psi(t)}$は
\begin{equation}
  \ket{\psi(t)} = U(t,t_0)\ket{\psi(t_0)}
  \label{time_evolution_eq1}
\end{equation}
と、時刻$t_0$からの時間発展として書くことができる。
(\ref{time_evolution_eq1})を(\ref{schrodinger_eq1})に戻すことによって、
\begin{equation}
  i\hbar \dfrac{\partial}{\partial t}U(t,t_0)\ket{\psi(t_0)} = HU(t,t_0)\ket{\psi(t_0)}
\end{equation}
また、$t = t_0$では状態は時間発展しないので、$U(t,t_0)$は
\begin{align}
  i\hbar \dfrac{\partial}{\partial t}U(t,t_0) &= HU(t,t_0)\label{schrodinger_eq2}\\
  U(t_0,t_0) &= \mbox{\boldmath $I$} \label{U1}
\end{align}
を満たさなければならない。$\mbox{\boldmath $I$}$は単位演算子である。
(\ref{schrodinger_eq2})を解くということは、schr\"{o}dinger方程式(\ref{schrodinger_eq1})を解くということである。
\begin{equation}
  U(t_2,t_0) = U(t_2.t_1)U(t_1,t_0), (t_2>t_1>t_0)
\end{equation}
を利用して、無限小時間$\Delta t$の時間発展$t_1 = t_0 + \Delta t$を考える。ハミルトニアン$H$が時間に依存するとき、
schr\"{o}dinger方程式(\ref{schrodinger_eq2})は、
\begin{equation}
  i\hbar \dfrac{U(t_1,t_0) - U(t_0,t_0)}{\Delta t} = H(t_1)U(t_0,t_0)
\end{equation}
より、
\begin{equation}
  U(t_1,t_0) = \left(1 - \dfrac{i}{\hbar}H(t_1)\Delta t\right)U(t_0,t_0)
\end{equation}
となる。また、次の時刻$t_2 = t_1 + \Delta t$では
\begin{equation}
    U(t_2,t_0) = \left(1 - \dfrac{i}{\hbar}H(t_2)\Delta t\right)\left(1 - \dfrac{i}{\hbar}H(t_1)\Delta t\right)U(t_0,t_0)
\end{equation}
となる。これを繰り返していくことによって、時刻$t_0$から$t$までの時間発展は、$\Delta t = \dfrac{t-t_0}{N}$、$t = t_N$、$t_k = t_0 + k\Delta t$とすれば、
\begin{equation}
  U(t_N,t_0) = \lim_{N\to \infty} \left(1 - \dfrac{i}{\hbar}H(t_N)\Delta t\right)\left(1 - \dfrac{i}{\hbar}H(t_{N-1})\Delta t\right)\cdots\left(1 - \dfrac{i}{\hbar}H(t_1)\Delta t\right)U(t_0,t_0)
\end{equation}
また、ハミルトニアン$H$が時間に依存しない場合は、指数関数の定義より、
\begin{equation}
    U(t_N,t_0) = \lim_{N\to \infty} \left(1 - \dfrac{i}{\hbar}H \Delta t\right)^N U(t_0,t_0) = e^{-i(t_N - t_0)H/\hbar}U(t_0,t_0)
    \label{U_exp}
\end{equation}
となる。
逆に、(\ref{U_exp})をテイラー展開すれば、
\begin{equation}
  U(t_1,t_0) = e^{-i(t_1 - t_0)H/\hbar}U(t_0,t_0) = 1 - \dfrac{i}{\hbar}\Delta t H + \mathcal{O}((\Delta t)^2)
\end{equation}
となるので、数値計算を行う上では$N$は有限なので、$\mathcal{O}((\Delta t)^2)$の誤差が出てくる。これはハミルトニアン$H$が時間に依存していても同様である。
また、ハミルトニアン$H$が$H=A+B$と運動エネルギー$A$とポテンシャルエネルギー$B$の和になっているとき、
\begin{equation}
  e^{-i(t_N - t_0)(A+B)/\hbar} = \lim_{N\to\infty}\left(e^{-i\Delta t A/\hbar} e^{-i\Delta t B/\hbar}\right)^N
\end{equation}
と書ける。これを{\bf Trotter公式}と呼ぶ。
\begin{comment}
$N$が有限の場合を考えると、
\begin{equation}
  e^{-i(t_N - t_0)(A+B)/\hbar} = \left(e^{-i\Delta t A/\hbar} e^{-i\Delta t B/\hbar}\right)^N + \mathcal{O}\left(\dfrac{(\Delta t)^2}{N}\right)
\end{equation}
\end{comment}
