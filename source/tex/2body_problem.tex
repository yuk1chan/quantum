\section{2体問題 - 重心運動の分離}
	シュレディンガー方程式は$1$個の粒子の運動を記述するものである。
	では、$2$つの粒子の場合にはどうすれば良いだろうか。
\subsection{古典力学での重心運動の分離}
$2$つの粒子の質量を$m_1, m_2$、位置を$\bm{r_1},\bm{r_2}$とする。この$2$粒子系のラグランジアン$L$は
\begin{equation}
	\label{L}
	L = \dfrac{m_1}{2}\dot{\bm{r_1}}^2 + \dfrac{m_2}{2}\dot{\bm{r_2}}^2 - V(|\bm{r_1} - \bm{r_2}|)
\end{equation}
となる。また、対応する運動量は
\begin{align}
	\bm{p_1} &= m_1\dot{\bm{r_1}} \\
	\bm{p_2} &= m_2\dot{\bm{r_2}}
\end{align}
となる。
ここで、$\bm{r_1},\bm{r_2}$の代わりに、
重心座標$\bm{R}$と相対座標$\bm{r}$を導入する。
\begin{align}
	\label{CM_R}
	\bm{R} &= \dfrac{m_1\bm{r_1} + m_2\bm{r_2}}{m_1 + m_2} \\
	\label{rel_r}
	\bm{r} &= \bm{r_1} - \bm{r_2}
\end{align}
これを用いて、
\begin{align}
	\bm{r_1} &= \bm{R} + \dfrac{m_2}{m_1 + m_2}\bm{r} \\
	\bm{r_2} &= \bm{R} - \dfrac{m_1}{m_1 + m_2}\bm{r}
\end{align}
これを(\ref{L})に代入すると、
\begin{align}
	L &= \dfrac{1}{2}m_1\left( \dot{\bm{R}} + \dfrac{m_2}{m_1 + m_2}\dot{\bm{r}} \right)^2
		+ \dfrac{1}{2}m_2\left( \dot{\bm{R}} - \dfrac{m_1}{m_1 + m_2}\dot{\bm{r}} \right)^2
		- V(r) \\
		&= \dfrac{1}{2}M\dot{\bm{R}}^2 + \dfrac{1}{2}\mu\dot{\bm{r}}^2 - V(r)
\end{align}
ここで、$M$を全質量、$\mu$を換算質量と呼び、
\begin{align}
	M &= m_1 + m_2 \\
	\mu &= \dfrac{m_1m_2}{m_1+m_2}
\end{align}
とする。
また、$\bm{R}$に対応する運動量$\bm{P}$を全運動量、$\bm{r}$に対応する運動量$\bm{p}$を相対運動量といい、
\begin{align}
	\label{CM_P}
	\bm{P} &= M\dot{\bm{R}} = m_1\dot{\bm{r_1}} + m_2\dot{\bm{r_2}} \\
	\label{rel_p}
	\bm{p} &= \mu\dot{\bm{r}} = \dfrac{m_2\dot{\bm{p_1}} - m_1\dot{\bm{p_2}}}{m_1 + m_2}
\end{align}
である。
これを用いることで、ハミルトニアン$H$は
\begin{equation}
	H = \dfrac{\bm{P}^2}{2M} + \dfrac{\bm{p}^2}{2\mu} + V(r)
\end{equation}
とすることができる。
ここで、各粒子の運動方程式は
\begin{align}
	m_1 \dfrac{d^2 \bm{r_1}}{d t^2} &= - \dfrac{\partial V}{\partial \bm{r_1}} = - \dfrac{\partial V}{\partial \bm{r}} \\
	m_2 \dfrac{d^2 \bm{r_2}}{d t^2} &= - \dfrac{\partial V}{\partial \bm{r_2}} = + \dfrac{\partial V}{\partial \bm{r}} \\
\end{align}
である。これより
\begin{align}
	\dot{\bm{P}} &= m_1\ddot{\bm{r_1}} + m_2\ddot{\bm{r_2}} = - \dfrac{\partial V}{\partial \bm{r}} + \dfrac{\partial V}{\partial \bm{r}} = 0 \\
	\dot{\bm{p}} &= \dfrac{1}{m_1 + m_2}(m_2 m_1 \ddot{\bm{r_1}} - m_1 m_2 \ddot{\bm{r_2}}) =
	\dfrac{1}{m_1 + m_2}\left( -m_2 \dfrac{\partial V}{\partial \bm{r}} - m_1\dfrac{\partial V}{\partial \bm{r}} \right)
	= -\dfrac{\partial V}{\partial \bm{r}}
\end{align}
よって、重心運動は直線運動、相対運動はポテンシャル$V(r)$内における質量$\mu$の粒子の運動であると分かる。
\footnote{$F = -\nabla V$を思い出そう。すると、$\dot{\bm{P}}$は$F = 0$なので、直線運動。$\dot{\bm{p}}$は$F = - \dfrac{\partial V}{\partial \bm{r}}$なので、ポテンシャル$V$内での運動。}

\subsection{量子系での重心運動の分離}\label{CM_separation}
古典での重心運動の分離は量子論でもできる。
$2$つの粒子の位置と運動量を表す演算子$\hat{\bm{r_1}},\hat{\bm{r_2}},\hat{\bm{p_1}},\hat{\bm{p_2}}$は
交換関係
\begin{equation}
	\label{communication_relation_r_p}
	[\hat{\bm{r_i}},\hat{\bm{p_j}}] = i\hbar\delta_{ij}
\end{equation}
を満たす。
(\ref{CM_R})、(\ref{rel_r})、(\ref{CM_P})、(\ref{rel_p})と同じように演算子$\hat{\bm{R}},\hat{\bm{r}},\hat{\bm{P}},\hat{\bm{p}}$を導入する。
交換関係(\ref{communication_relation_r_p})を用いて$[\hat{\bm{R}},\hat{\bm{P}}]$と$[\hat{\bm{r}},\hat{\bm{p}}]$を計算する。
\begin{align}
	[\hat{\bm{R}},\hat{\bm{P}}]
	&= \dfrac{m_1\bm{r_1} + m_2\bm{r_2}}{m_1 + m_2}(\bm{p_1} + \bm{p_2}) - (\bm{p_1} + \bm{p_2})\dfrac{m_1\bm{r_1} + m_2\bm{r_2}}{m_1 + m_2}\\
	&= \dfrac{1}{m_1 + m_2}[m_1\bm{r_1}\bm{p_1} + m_1\bm{r_1}\bm{p_2} + m_2\bm{r_2}\bm{p_1} + m_2\bm{r_2}\bm{p_2}
	- m_1\bm{p_1}\bm{r_1} - m_2\bm{p_1}\bm{r_2} - m_1\bm{p_2}\bm{r_1} - m_2\bm{p_2}\bm{r_2}] \\
	&= \dfrac{1}{m_1 + m_2}[ m_1(\bm{r_1}\bm{p_1} - \bm{p_1}\bm{r_1}) + m_2(\bm{r_2}\bm{p_2} - \bm{p_2}\bm{r_2})] \\
	&= i\hbar
\end{align}
\begin{align}
	[\hat{\bm{r}},\hat{\bm{p}}]
	&= (\bm{r_1} - \bm{r_2})\dfrac{m_2\bm{p_1} - m_1\bm{p_2}}{m_1 + m_2} - \dfrac{m_2\bm{p_1} - m_1\bm{p_2}}{m_1 + m_2}(\bm{r_1} - \bm{r_2}) \\
	&= \dfrac{1}{m_1 + m_2}[m_2\bm{r_1}\bm{p_1} - m_1\bm{r_1}\bm{p_2} - m_2\bm{r_2}\bm{p_1} + m_1\bm{r_2}\bm{p_2}
	- m_2\bm{p_1}\bm{r_1} + m_2\bm{p_1}\bm{r_2} + m_1\bm{p_2}\bm{r_1} - m_1\bm{p_2}\bm{r_2}] \\
	&= \dfrac{1}{m_1 + m_2}[m_2(\bm{r_1}\bm{p_1} - \bm{p_1}\bm{r_1}) + m_1(\bm{r_2}\bm{p_2} - \bm{p_2}\bm{r_2})] \\
	&= i\hbar
\end{align}
と交換関係が成立する。

$2$粒子系のハミルトニアン$\hat{H}$は
\begin{equation}
	\hat{H} = \dfrac{\hat{\bm{p_1}}^2}{2m_1} + \dfrac{\hat{\bm{p_2}}^2}{2m_2} + V(|\bm{r_1} - \bm{r_2}|)
\end{equation}
と書けて、$\hat{\bm{R}},\hat{\bm{r}},\hat{\bm{P}},\hat{\bm{p}}$を用いると、これらの演算子の交換関係から
\begin{equation}
\hat{H} = \dfrac{\hat{\bm{P}}^2}{2M} + \dfrac{\hat{\bm{p}}^2}{2\mu} + V(r)
\end{equation}
と書ける。
ここで、
\begin{align}
	\hat{H}_{CM} &= \dfrac{\hat{\bm{P}}^2}{2M} \\
	\hat{H}_{rel} &= \dfrac{\hat{\bm{p}}^2}{2\mu} + V(r)
\end{align}
とすると、\footnote{CM:center of mass (重心) \\ rel:relative(相対運動)}
\begin{equation}
	\hat{H} = \hat{H}_{CM} + \hat{H}_{rel}
\end{equation}
となり、交換関係(\ref{communication_relation_r_p})があるので、
\begin{equation}
	[\hat{H}_{CM},\hat{H}_{rel}] = 0
\end{equation}

重心座標と相対座標で表したシュレディンガー方程式
\begin{equation}
	(\hat{H}_{CM} + \hat{H}_{rel})\psi(\bm{R},\bm{r}) = E\psi(\bm{R},\bm{r})
\end{equation}
の解を考える。
このシュレディンガー方程式は変数分離形である。
\begin{equation}
	\psi(\bm{R},\bm{r}) = \psi_{CM}(\bm{R})\psi_{rel}(\bm{r})
\end{equation}
と分離して元に戻すと、
\begin{equation}
	\hat{H}_{CM}\psi_{CM}(\bm{R})\psi_{rel}(\bm{r}) + \hat{H}_{rel}\psi_{CM}(\bm{R})\psi_{rel}(\bm{r}) = E\psi_{CM}(\bm{R})\psi_{rel}(\bm{r})
\end{equation}
両辺を$\psi_{CM}(\bm{R})\psi_{rel}(\bm{r})$で割ると、
\begin{equation}
	\dfrac{1}{\psi_{CM}}\hat{H}_{CM}\psi_{CM} + \dfrac{1}{\psi_{rel}}\hat{H}_{rel}\psi_{rel} = E
\end{equation}
ここで、左辺第$1$項目は$R$に関して、第$2$項目は$r$に関して、右辺は定数$E$である。
つまり、左辺の各項は定数である必要がある。
\begin{align}
	\dfrac{1}{\psi_{CM}}\hat{H}_{CM}\psi_{CM} &= E_{CM} \\
	\dfrac{1}{\psi_{rel}}\hat{H}_{rel}\psi_{rel} &= E_{rel}
\end{align}
よって、
\begin{align}
	\hat{H}_{CM}\psi_{CM} &= E_{CM}\psi_{CM} \\
	\hat{H}_{rel}\psi_{rel} &= E_{rel}\psi_{rel}
\end{align}
つまり、
\begin{align}
	\label{CM_schrodinger_eq}
	-\dfrac{\hbar^2}{2M}\Delta_R\psi_{CM}(\bm{R}) &= E_{CM}\psi_{CM}(\bm{R}) \\
	\label{rel_schrodinger_eq}
	\left[ -\dfrac{\hbar^2}{2\mu}\Delta_r + V(\bm{r})\right] &= E_{rel}\psi_{rel}(\bm{r})
\end{align}
ここで、$\Delta_R$は$R$座標に関するラプラシアン で、$\Delta_r$は$r$座標に関するラプラシアンである。

(\ref{CM_schrodinger_eq})は質量$M$の自由粒子のシュレディンガー方程式であり、$2$粒子系の重心の運動を表している。
解である$\psi_R$は平面波である。$E_R$は重心の運動のエネルギー固有値。

(\ref{rel_schrodinger_eq})は質量$\mu$の$1$つ粒子が中心力ポテンシャル$V(\bm{r})$内で運動する時のシュレディンガー方程式である。
つまり、$2$体問題は換算質量を用いることで$1$体問題に帰着することができる。$E_{rel}$は相対運動のエネルギー固有値。
